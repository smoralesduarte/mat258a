\documentclass[11pt]{article}

\usepackage[utf8]{inputenc}
\usepackage{amsmath, amssymb, amsthm}
\usepackage{geometry}
\usepackage{graphicx}
\usepackage{enumitem}
\usepackage{xcolor}
\usepackage{hyperref}
\usepackage{tikz}
\usepackage{fancyhdr}
\usepackage{biblatex}


\geometry{letterpaper, margin=1in}
\pagestyle{fancy}
\fancyhf{}
\rhead{MATH 258A — Spring 2025}
\lhead{Santiago Morales}
\rfoot{Page \thepage}

\title{\vspace{-1cm} \textbf{Homework 3, MAT 258A}}
\author{Santiago Morales \\ UC Davis \\ \texttt{moralesduarte@ucdavis.edu}}
\date{\today}

\begin{document}

\maketitle
\vspace{-1em}
\hrule
\vspace{1em}

\paragraph{1.  KKT conditions at $(4,2)$ show non-optimality.}
Write the constraints in the form
\[
g_1(x)=x_1-x_2-2\le 0,\qquad
g_2(x)=-x_1\le 0,\qquad
g_3(x)=-x_2\le 0 .
\]
For a maximisation problem the Lagrangian is
\(L(x,\lambda)=f(x)-\sum_{i=1}^3\lambda_i g_i(x)\)
with $\lambda_i\ge 0$.  
The objective is
\(f(x)=x_1/(x_2+1)\) and its gradient
\(
\nabla f(x)=\bigl((x_2+1)^{-1},\,-x_1\,(x_2+1)^{-2}\bigr).
\)
At the candidate point $(4,2)$ the active set is
$g_1=0$; $g_2,g_3<0$.  
Stationarity therefore requires
\[
\nabla f(4,2)-\lambda_1\nabla g_1(4,2)=0
\Longrightarrow
(1/3,\,-4/9)-\lambda_1(1,-1)=0,
\]
which yields \(\lambda_1=1/3\) from the first component and 
\(\lambda_1=-4/9\) from the second—a contradiction.  
Hence $(4,2)$ violates KKT and cannot be optimal.

\paragraph{2.  A point satisfying the KKT conditions.}
Because the denominator $x_2+1$ should be as small as possible
while the numerator $x_1$ should be as large as possible, optimality
must occur on the \emph{boundary} $x_1-x_2=2$ with $x_2$ minimal,
i.e.\ $x_2=0,\;x_1=2$.

At $(2,0)$ we have $g_1=g_3=0,\,g_2<0$.  
With multipliers $\lambda_1=\lambda_3=1,\;\lambda_2=0$,
\[
\nabla f(2,0)-\lambda_1\nabla g_1-\lambda_3\nabla g_3
=(1,-2)-(1,-1)-(0,-1)=(0,0),
\]
and complementary slackness holds.
Thus $(2,0)$ satisfies all KKT conditions.

\paragraph{3.  The problem is \emph{not} convex.}
A maximisation problem is convex only when the objective is \emph{concave} on a
convex feasible set.  The feasible region here is a simplex and is therefore
convex; the issue is the objective 
\(f(x)=x_{1}/(x_{2}+1)\).

Take 
\(A=(0,2)\)  and  \(B=(2,0)\),
both feasible.
Their midpoint is
\(C=(1,1)\).
Now
\[
f(A)=\frac{0}{3}=0,\quad
f(B)=\frac{2}{1}=2,\quad
f(C)=\frac{1}{2}=0.5 .
\]
Concavity would demand
\(
f(C) \;\ge\; \tfrac12\,f(A)+\tfrac12\,f(B) = 1.
\)
But \(f(C)=0.5<1\), violating the concavity inequality.
Hence \(f\) is \emph{not} concave on the feasible region,
so the optimisation problem is \emph{non-convex}. 

\paragraph{4.  Optimal solution.}
On the active face $x_1-x_2=2$ we have
\(f= (x_2+2)/(x_2+1)=1+1/(x_2+1)\), which is strictly
decreasing in $x_2\ge0$.  Hence the maximum is attained
at the minimal admissible $x_2$, namely $x_2^\star=0$,
with $x_1^\star=2$.  Therefore
\[
\boxed{\;x^\star=(2,0),\; f^\star=2\;}.
\]

(Joint work with Ian Gallagher)
\end{document}